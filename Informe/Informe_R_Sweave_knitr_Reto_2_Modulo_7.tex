\documentclass{article}

\documentclass{article}
\usepackage{graphicx}
\usepackage{hyperref}

\begin{Schunk}
\begin{Sinput}
> knitr::opts_chunk$set(echo = TRUE)
> library(readxl)
> library(dplyr)
> library(openxlsx)
> library(tidyverse)
> library(xfun)
\end{Sinput}
\end{Schunk}

\usepackage{Sweave}
\begin{document}
\Sconcordance{concordance:Informe_R_Sweave_knitr_Reto_2_Modulo_7.tex:Informe_R_Sweave_knitr_Reto_2_Modulo_7.Rnw:%
1 5 1 1 2 1 0 5 1 3 0 1 2 1 1 1 0 9 1 1 2 1 0 1 2 4 0 1 2 2 1 1 5 4 0 1 %
6 5 0 1 6 8 0 1 2 2 1 1 3 2 0 1 1 3 0 1 2 4 1 1 3 2 0 1 5 7 0 1 2 2 1 1 %
5 9 0 1 4 1 1}


\title{Informe de empleo industrial y SDI}
\author{Alexandra Reguera González}
\date{\today}
\maketitle

\section{Cargar y preparar los datos}

\begin{Schunk}
\begin{Sinput}
> datos <- read_excel("../Datos/industry_workers_percent_of_employment.xlsx")
> datos_largos <- datos %>%
+   pivot_longer(cols = -country, names_to = "year", values_to = "employment_rate")
\end{Sinput}
\end{Schunk}

\section{Análisis descriptivo}

\begin{Schunk}
\begin{Sinput}
> # Tasa de empleo global por año
> global_employment_rate <- datos_largos %>%
+   group_by(year) %>%
+   summarise(mean_employment_rate = mean(employment_rate, na.rm = TRUE))
> # Países que aumentaron su tasa de empleo:
> employment_increase <- datos_largos %>%
+   group_by(country) %>%
+   arrange(year) %>%
+   summarise(change_in_employment = employment_rate[n()] - employment_rate[1]) %>%
+   filter(change_in_employment > 0)
> # Países que redujeron su tasa de empleo
> employment_decrease <- datos_largos %>%
+   group_by(country) %>%
+   arrange(year) %>%
+   summarise(change_in_employment = employment_rate[n()] - employment_rate[1]) %>%
+   filter(change_in_employment < 0)
\end{Sinput}
\end{Schunk}

\section{Análisis de correlación}

\begin{Schunk}
\begin{Sinput}
> datos_wide <- datos_largos %>%
+   pivot_wider(names_from = country, values_from = employment_rate)
> correlation_matrix <- cor(datos_wide[,-1], use = "pairwise.complete.obs")
\end{Sinput}
\end{Schunk}

\section{Visualización}

\subsection{Tasa de empleo global a lo largo del tiempo}

\begin{Schunk}
\begin{Sinput}
> # Convertir el año a numérico
> global_employment_rate$year <- as.numeric(global_employment_rate$year)
> # Crear la gráfica
> ggplot(global_employment_rate, aes(x = year, y = mean_employment_rate, group=1)) +
+   geom_line() +
+   labs(title = "Tasa de empleo global a lo largo del tiempo",
+        x = "Año", y = "Tasa de empleo global")
\end{Sinput}
\end{Schunk}

\subsection{Tasa de empleo promedio por continente a lo largo del tiempo}

\begin{Schunk}
\begin{Sinput}
> # Definir los países para cada continente
> # [código omitido por razones de longitud]
> # Crear una nueva columna 'continente' con NA (sin asignación inicial)
> datos_largos$continente <- NA
> # Luego asignar los continentes a cada país
> # [código omitido por razones de longitud]
\end{Sinput}
\end{Schunk}

\end{document}
